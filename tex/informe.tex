\documentclass[a4paper, 10pt, twoside]{article}

\usepackage[top=1in, bottom=1in, left=1in, right=1in]{geometry}
\usepackage[utf8]{inputenc}
\usepackage[spanish, es-ucroman, es-noquoting]{babel}
\usepackage{setspace}
\usepackage{fancyhdr}
\usepackage{lastpage}
\usepackage{amsmath}
\usepackage{amsfonts}
\usepackage{amsthm}
\usepackage{verbatim}
\usepackage{graphicx}
\usepackage{float}
\usepackage[noend]{algpseudocode}
\usepackage{enumitem} % Provee macro \setlist
\usepackage[toc, page]{appendix}
\usepackage{amsthm}
\usepackage{epstopdf}
\usepackage{amssymb}
\usepackage{caption}
\usepackage{subcaption}

%%%%%%%%%% Configuración de amsthm %%%%%%%%%%

\newtheorem{propiedad}{Propiedad}

%%%%%%%%%% Configuración de Fancyhdr - Inicio %%%%%%%%%%
\pagestyle{fancy}
\thispagestyle{fancy}
\lhead{Trabajo Práctico 3 · Algoritmos y Estructuras de Datos III}
\rhead{Aboy · Almansi · Canay · Decroix}
\renewcommand{\footrulewidth}{0.4pt}
\cfoot{\thepage /\pageref{LastPage}}

\fancypagestyle{caratula} {
   \fancyhf{}
   \cfoot{\thepage /\pageref{LastPage}}
   \renewcommand{\headrulewidth}{0pt}
   \renewcommand{\footrulewidth}{0pt}
}
%%%%%%%%%% Configuración de Fancyhdr - Fin %%%%%%%%%%


%%%%%%%%%% Configuración de Algorithmic - Inicio %%%%%%%%%%
% Entorno propio para customizar la presentación del pseudocódigo
\newenvironment{pseudo}[1][]{%
    \vspace{0.5em}%
    \begin{algorithmic}%
}
{%
    \end{algorithmic}%
    \vspace{0.5em}%
}

% Valores de verdad
\newcommand{\True}{\textbf{true}}
\newcommand{\False}{\textbf{false}}

% Conectivo 'in' para usar así: \ForAll{$foo$ \In $bar$}
\newcommand{\In}{\textbf{in} }

% Conectivo 'to' para usar así: \For{$i = 1$ \In $n$}
\newcommand{\To}{\textbf{to} }

% Complejidades
\newcommand{\Ode}[1]{\hfill $O(#1)$}
%%%%%%%%%% Configuración de Algorithmic - Fin %%%%%%%%%%


%%%%%%%%%% Miscelánea - Inicio %%%%%%%%%%
% Evita que el documento se estire verticalmente para ocupar el espacio vacío
% en cada página.
\raggedbottom

% Deshabilita sangría en la primer línea de un párrafo.
\setlength{\parindent}{0em}

% Separación entre párrafos.
\setlength{\parskip}{0.5em}

% Separación entre elementos de listas.
\setlist{itemsep=0.5em}

% Asigna la traducción de la palabra 'Appendices'.
\renewcommand{\appendixtocname}{Apéndices}
\renewcommand{\appendixpagename}{Apéndices}
%%%%%%%%%% Miscelánea - Fin %%%%%%%%%%


%%%%%%%%%% Gráficos - Inicio %%%%%%%%%%
% Macro para incluir tres gráficos (dentro de una figura) de manera que
% entren todos en una sola página.
\newcommand{\tresgraficos}[3]{
    \newcommand{\separacion}{-2.2em}
    \vspace{\separacion}
    \include{#1}
    \vspace{\separacion}
    \include{#2}
    \vspace{\separacion}
    \include{#3}
}
%%%%%%%%%% Gráficos - Fin %%%%%%%%%%


\begin{document}


%%%%%%%%%%%%%%%%%%%%%%%%%%%%%%%%%%%%%%%%%%%%%%%%%%%%%%%%%%%%%%%%%%%%%%%%%%%%%%%
%% Carátula                                                                  %%
%%%%%%%%%%%%%%%%%%%%%%%%%%%%%%%%%%%%%%%%%%%%%%%%%%%%%%%%%%%%%%%%%%%%%%%%%%%%%%%

\thispagestyle{caratula}

\begin{center}

\includegraphics[width=0.6\textwidth]{./img/DC.jpg} 
\hfill

\vspace{2cm}

\begin{Huge}
Trabajo Práctico 3
\end{Huge}

\vspace{0.5cm}

\begin{Large}
Algoritmos y Estructuras de Datos III
\end{Large}

\vspace{1cm}

\begin{Large}
Primer Cuatrimestre de 2014
\end{Large}

\vspace{2cm}

\begin{tabular}{|c|c|c|}
\hline
Alumno & LU & E-mail\\
\hline
Aboy Solanes, Santiago    & 175/12 & santiaboy2@hotmail.com\\
Almansi, Emilio Guido     & 674/12 & ealmansi@gmail.com\\
Canay, Federico José      & 250/12 & fcanay@hotmail.com\\
Decroix, Facundo Nicolás  & 842/11 & fndecroix92@hotmail.com\\
\hline
\end{tabular}

\vspace{4cm}

Departamento de Computación,\\
Facultad de Ciencias Exactas y Naturales,\\
Universidad de Buenos Aires

\end{center}

\newpage


%%%%%%%%%%%%%%%%%%%%%%%%%%%%%%%%%%%%%%%%%%%%%%%%%%%%%%%%%%%%%%%%%%%%%%%%%%%%%%%
%% Índice                                                                    %%
%%%%%%%%%%%%%%%%%%%%%%%%%%%%%%%%%%%%%%%%%%%%%%%%%%%%%%%%%%%%%%%%%%%%%%%%%%%%%%%

\tableofcontents

\newpage


%%%%%%%%%%%%%%%%%%%%%%%%%%%%%%%%%%%%%%%%%%%%%%%%%%%%%%%%%%%%%%%%%%%%%%%%%%%%%%%
%% Introducción                                                              %%
%%%%%%%%%%%%%%%%%%%%%%%%%%%%%%%%%%%%%%%%%%%%%%%%%%%%%%%%%%%%%%%%%%%%%%%%%%%%%%%

\section{Introducción}
\label{sec:introduccion}
\subsection{Descripción del problema: Camino acotado de costo mínimo (CACM)}
\label{sub:introduccion-descripcion}

El objetivo de este trabajo práctico es resolver el problema de \emph{Camino acotado de costo mínimo} (CACM, a partir de ahora). Dado un grafo simple $G = (V,E)$, $w_1$ y $w_2$ funciones de peso, y $K$ $\in$ $\mathbb{R}_+$ constante, este problema consiste en hallar un camino $P$ de $u$ a $v$ ($u$, $v$ $\in$ $V$) con costo $w_1$(P) $\leq$ $K$ de manera tal que el costo $w_2$(P) sea mínimo.

\subsection{Ejemplos de problemas reales}
\label{sub:introduccion-ejemplos}

A continuación vamos a enumerar diversos ejemplos de la vida real que pueden modelarse utilizando CACM.

Un trabajador desea viajar a su oficina para trabajar utilizando el excelentísimo transporte público, y quiere llegar lo más rápido posible utilizando a lo sumo 10 patacones. En este caso, el problema se podría modelar con un grafo $G$ = ($V$,$E$) donde los nodos son las estaciones del transporte, y dos nodos $u$, $v$ $\in$ $V$ son adyacentes si dicho transporte viaja entre dichas estaciones. La función de peso $w_1$ devuelve la plata que cuesta ir de una parada a otra, $w_2$ devuelve el tiempo de viaje entre las estaciones y $K$ es igual a 10 patacones.

Un viajante desea ir de una ciudad A a una ciudad B, en un país donde hay una cabina de peaje en cada ruta que conecta dos ciudades. Este viajante desea ir de A a B, pagando lo menos posible tardando a lo sumo 2 horas. En este caso, el problema se podría modelar con un grafo $G$ = ($V$,$E$) donde los nodos son las ciudades, y dos nodos $u$, $v$ $\in$ $V$ son adyacentes si existe una ruta entre dichas ciudades. La función de peso $w_1$ devuelve el tiempo de viaje que se tarda en tomar esa ruta, $w_2$ devuelve la plata que cuesta el peaje de dicha ruta y $K$ es igual a 2 horas.

Otro viajante (al que le no le gusta ver el paisaje) desea viajar de la ciudad Ave a la ciudad Batracio minimizando el tiempo de viaje, sin viajar más de 200 km. En este último caso, el problema se podría modelar con un grafo $G$ = ($V$,$E$) donde los nodos son las ciudades, y dos nodos $u$, $v$ $\in$ $V$ son adyacentes si existe una ruta entre $u$ y $v$. La función de peso $w_1$ devuelve los kilómetros de la ruta, $w_2$ devuelve tiempo de viaje que se tarda en tomar esa ruta y $K$ es igual a 200 km.

\subsection{Aclaraciones}

Sea un grafo $G$=($V$,$E$), en este trabajo práctico, vamos a utilizar:
\begin{itemize}
 \item $n$ como la cantidad de nodos del grafo G.
 \item $m$ como la cantidad de aristas del grafo G.
 \item Cuando hablamos de ``nuestro algoritmo GRASP'', nos referimos a nuestro algoritmo que utiliza la metaheurística GRASP.
\end{itemize}


\newpage


%%%%%%%%%%%%%%%%%%%%%%%%%%%%%%%%%%%%%%%%%%%%%%%%%%%%%%%%%%%%%%%%%%%%%%%%%%%%%%%
%% Consideraciones                                                           %%
%%%%%%%%%%%%%%%%%%%%%%%%%%%%%%%%%%%%%%%%%%%%%%%%%%%%%%%%%%%%%%%%%%%%%%%%%%%%%%%

\section{Consideraciones}
\label{sec:consideraciones}
\subsection{Lenguaje de implementación}

Para implementar los diferentes algoritmos propuestos utilizamos el lenguaje C++, el cual presenta una serie de características muy convenientes. Este lenguaje es imperativo, al igual que el lenguaje de pseudocódigo utilizado para describir las soluciones y probar su correctitud. Adicionalmente, el mismo posee librerías estándar muy completas, versátiles y bien documentadas, lo cual permite abstraer el manejo de memoria, la implementación de estructuras de datos y algoritmos de uso frecuente, y provee mecanismos para realizar mediciones de tiempo de manera fidedigna.

\subsection{Mediciones de tiempo}
\label{subsec:mediciones-de-tiempo}

En las diferentes etapas de experimentación, llevamos a cabo mediciones de rendimiento sobre las implementaciones desarrolladas, midiendo el tiempo consumido para resolver diferentes instancias según el caso. Nos aseguramos de medir exclusivamente el tiempo consumido por la etapa de resolución, ignorando tareas adicionales propias al proceso como, por ejemplo, la generación de la instancia a ser resuelta.

La función del sistema que se escogió para medir intervalos de tiempo es la siguiente:

\begin{verbatim}
  int clock_gettime(clockid_t clk_id, struct timespec *tp);
\end{verbatim}

de la librería \emph{time.h}. La misma nos permite realizar mediciones de alta resolución, específicas al tiempo de ejecución del proceso que la invoca (y no al sistema en su totalidad), configurando el parámetro clk\_id con el valor CLOCK\_PROCESS\_CPUTIME\_ID\footnote{http://linux.die.net/man/3/clock\_gettime}.

Por otro lado, dado que la medición de tiempos en un sistema operativo activo introduce inherentemente un cierto nivel de ruido, la medición sobre cada instancia se realizó múltiples veces. Una vez obtenidos los distintos valores para una misma medición (es decir, para diferentes instancias del mismo tamaño), registramos como valor definitivo la mediana de la serie de valores. Escogimos este criterio en vez de, por ejemplo, tomar la media, ya que utilizar la mediana es menos susceptible a la presencia de valores atípicos o \emph{outliers}.

\subsection{Generación de instancias aleatorias}
\label{subsec:generacion-instancias-aleatorias}

Para la etapa de experimentación, desarrollamos un generador de instancias aleatorias del problema CACM. El mismo recibe como parámetros los valores $n, m, max_{w_1}, max_{w_2}, K$, y genera una instancia conteniendo un grafo $G$ con las siguientes características:

\begin{itemize}
  \item $\#V(G) = n$.
  \item $\#E(G) = m$.
  \item $(\forall e \in E(G))\,w_1(e) \leq max_{w_1} \land w_2(e) \leq max_{w_2}$.
\end{itemize}

Se definen adicionalmente los nodos origen y destino $u$ y $v$, y se establece a $K$ como la cota para el peso del camino según $w_1$.

\newpage


%%%%%%%%%%%%%%%%%%%%%%%%%%%%%%%%%%%%%%%%%%%%%%%%%%%%%%%%%%%%%%%%%%%%%%%%%%%%%%%
%% Algoritmo exacto                                                          %%
%%%%%%%%%%%%%%%%%%%%%%%%%%%%%%%%%%%%%%%%%%%%%%%%%%%%%%%%%%%%%%%%%%%%%%%%%%%%%%%

\section{Algoritmo exacto}
\label{sec:algoritmo-exacto}

  \subsection{Desarrollo del algoritmo}
  \label{sub:algoritmo-exacto-desarrollo}
  \input{./exacto/desarrollo.tex}

  \subsection{Complejidad temporal de peor caso}
  \label{sub:algoritmo-exacto-complejidad}
  \input{./exacto/complejidad.tex}

  \subsection{Experimentación}
  \label{sub:algoritmo-exacto-experimentacion}
  \input{./exacto/experimentacion.tex}

\newpage


%%%%%%%%%%%%%%%%%%%%%%%%%%%%%%%%%%%%%%%%%%%%%%%%%%%%%%%%%%%%%%%%%%%%%%%%%%%%%%%
%% Algoritmos heurísticos                                                    %%
%%%%%%%%%%%%%%%%%%%%%%%%%%%%%%%%%%%%%%%%%%%%%%%%%%%%%%%%%%%%%%%%%%%%%%%%%%%%%%%

\section{Algoritmos heurísticos}
\label{sec:algoritmos-heuristicos}

  \subsection{Heuristica constructiva golosa}
  \label{sub:algoritmos-heuristicos-goloso}
      \subsubsection{Desarrollo del algoritmo}
      \label{subsub:algoritmos-heuristicos-goloso-desarrollo.tex}
      \input{./goloso/desarrollo.tex}

      \subsubsection{Complejidad temporal de peor caso}
      \label{subsub:algoritmos-heuristicos-goloso-complejidad.tex}
      \input{./goloso/complejidad.tex}

      \subsubsection{Análisis de la calidad de las soluciones}
      \label{subsub:algoritmos-heuristicos-goloso-calidad.tex}
      \input{./goloso/calidad.tex}

      \subsubsection{Experimentación}
      \label{subsub:algoritmos-heuristicos-goloso-experimentacion.tex}
      \input{./goloso/experimentacion.tex}
  \newpage

  \subsection{Heuristica de busqueda local}
  \label{sub:algoritmos-heuristicos-busqueda}
      
      \label{subsub:algoritmos-heuristicos-busqueda-introduccion_busqueda.tex}
      \input{./busqueda/introduccion_busqueda.tex}
            
      \subsubsection{Vecindad}
      \label{subsub:algoritmos-heuristicos-busqueda-desarrollo_vecindad.tex}
      \input{./busqueda/desarrollo_vecindad.tex}

      \subsubsection{Desarrollo del algoritmo}
      \label{subsub:algoritmos-heuristicos-busqueda-desarrollo_algoritmo.tex}
      \input{./busqueda/desarrollo_algoritmo.tex}
      
      \subsubsection{Complejidad temporal de peor caso}
      \label{subsub:algoritmos-heuristicos-busqueda-complejidad.tex}
      \input{./busqueda/complejidad.tex}

      \subsubsection{Análisis de la calidad de las soluciones}
      \label{subsub:algoritmos-heuristicos-busqueda-calidad.tex}
      \input{./busqueda/calidad.tex}

      \subsubsection{Experimentación}
      \label{subsub:algoritmos-heuristicos-busqueda-experimentacion.tex}
      \input{./busqueda/experimentacion.tex}
  \newpage

  \subsection{Heuristica GRASP}
  \label{sub:algoritmos-heuristicos-grasp}
      \subsubsection{Desarrollo del algoritmo}
      \label{subsub:algoritmos-heuristicos-grasp-desarrollo.tex}
      \input{./grasp/desarrollo.tex}

      \subsubsection{Complejidad temporal de peor caso}
      \label{subsub:algoritmos-heuristicos-grasp-complejidad.tex}
      \input{./grasp/complejidad.tex}

      \subsubsection{Experimentación}
      \label{subsub:algoritmos-heuristicos-grasp-experimentacion.tex}
      \input{./grasp/experimentacion.tex}

\newpage


%%%%%%%%%%%%%%%%%%%%%%%%%%%%%%%%%%%%%%%%%%%%%%%%%%%%%%%%%%%%%%%%%%%%%%%%%%%%%%%
%% Comparación entre algoritmos y experimentación general                    %%
%%%%%%%%%%%%%%%%%%%%%%%%%%%%%%%%%%%%%%%%%%%%%%%%%%%%%%%%%%%%%%%%%%%%%%%%%%%%%%%

\section{Comparación entre algoritmos y experimentación general}
\label{sec:experimentacion-general}
\input{experimentacion-general.tex}

\newpage

%%%%%%%%%%%%%%%%%%%%%%%%%%%%%%%%%%%%%%%%%%%%%%%%%%%%%%%%%%%%%%%%%%%%%%%%%%%%%%%
%% Conclusión                                                                %%
%%%%%%%%%%%%%%%%%%%%%%%%%%%%%%%%%%%%%%%%%%%%%%%%%%%%%%%%%%%%%%%%%%%%%%%%%%%%%%%

\section{Conclusión}
\label{sec:conclusion}
Para concluir, nos parece importante recalcar que ninguna heurística que planteamos es estrictamente mejor que todas las demás. Sin embargo, podemos asegurar que nuestro algoritmo de búsqueda local es siempre mejor o igual que nuestro algoritmo goloso, debido a por como está implementado. 

Otra punto que concluimos es que la diferencia entre nuestras 3 heurísticas, en general, no es grande. Creemos que esto se debe a que la vecindad propuesta no es muy bueno. Esto genera pocos cambios en la búsqueda local por lo que todas las heurísticas se parecen al algoritmo goloso. Una posible solución a esto es ampliar nuestra vecindad, generando las mismas operaciones pero en vez de a un nodo, de a $k$.

Esto tiene sentido, ya que una heurística es una técnica para resolver problemas donde la solución dada puede no ser óptima. Si pudiéramos dar la solución óptima	 en una cantidad de tiempo lo suficientemente chico, no tendría sentido hablar de heurísticas.

\newpage

%%%%%%%%%%%%%%%%%%%%%%%%%%%%%%%%%%%%%%%%%%%%%%%%%%%%%%%%%%%%%%%%%%%%%%%%%%%%%%%
%% Apéndices                                                                 %%
%%%%%%%%%%%%%%%%%%%%%%%%%%%%%%%%%%%%%%%%%%%%%%%%%%%%%%%%%%%%%%%%%%%%%%%%%%%%%%%

\begin{appendices}

\section{Algoritmo exacto}
\label{exacto-codigo}
\verbatiminput{./codigo-fuente/exacto.cpp}

\newpage

\section{Algoritmo de Dijkstra randomizado}
\label{dijkstra-codigo}

\verbatiminput{./codigo-fuente/dijkstra.cpp}

\newpage

\section{Algoritmo goloso}
\label{goloso-codigo}
\verbatiminput{./codigo-fuente/goloso.cpp}

\newpage

\section{Algoritmo búsqueda local}
\label{busqueda-local-codigo}
\verbatiminput{./codigo-fuente/busqueda.cpp}

\newpage

\section{Algoritmo tipo GRASP}
\label{grasp-codigo}
\verbatiminput{./codigo-fuente/grasp.cpp}

\newpage

\section{Gráficos extra}
\label{graficos-extra}
\begin{figure}[H]
    \begin{minipage}{0.5\linewidth}
      \includegraphics[width=\linewidth]{graficos/todos_tiempo_randm4K12.eps}
      \caption{Tiempo grafos ralos aleatorios, cota holgada}\label{fig:extra1}
    \end{minipage}
    \hfill
    \begin{minipage}{0.5\linewidth}
      \includegraphics[width=\linewidth]{graficos/todos_tiempo_randm4K12SinG.eps}
      \caption{Ídem sin Grasp}\label{fig:extra2}
    \end{minipage}    
\end{figure}

\begin{figure}[H]
    \begin{minipage}{0.5\linewidth}
      \includegraphics[width=\linewidth]{graficos/todos_tiempo_randm8K2.eps}
      \caption{Tiempo grafos densos aleatorios, cota ajustada}\label{fig:extra3}
    \end{minipage}
    \hfill
    \begin{minipage}{0.5\linewidth}
      \includegraphics[width=\linewidth]{graficos/todos_tiempo_randm8K2SinG.eps}
      \caption{Ídem sin Grasp}\label{fig:extra4}
    \end{minipage}    
\end{figure}

\begin{figure}[H]
    \begin{minipage}{0.5\linewidth}
      \includegraphics[width=\linewidth]{graficos/todos_tiempo_randm8K12.eps}
      \caption{Calidad grafos densos aleatorios, cota holgada}\label{fig:extra5}
    \end{minipage}
    \hfill
    \begin{minipage}{0.5\linewidth}
      \includegraphics[width=\linewidth]{graficos/todos_tiempo_randm8K12SinG.eps}
      \caption{Ídem sin Grasp}\label{fig:extra6}
    \end{minipage}    
\end{figure}

\end{appendices}

\end{document}